\documentclass{aip-cp}
\usepackage[numbers]{natbib}
\usepackage[T1]{fontenc}
\usepackage{rotating}
\usepackage{url} 
\usepackage{graphicx}
\usepackage[dvipsnames]{xcolor}
\usepackage{soul}
\newtheorem{definition}{Definition}[section] 
\newtheorem{theorem}{Theorem}[section] 
\newtheorem{Example}{Example}[section] 
\newtheorem{Simulation}{Simulation}[section] 

% Document starts
\begin{document}
 
% Title portion
\title{xx}

%%
\author[aff1]{J.A. de Almeida Schultz}
\eaddress{zeadolfo@gmail.com}

\author[aff1]{V.A. Gonz\'{a}lez-L\'{o}pez}
\eaddress{veronica@ime.unicamp.br}

\affil[aff1]{University of Campinas - Department of Statistics. Rua S\'{e}rgio Buarque de Holanda 651. Campinas, S.P. Brazil. CEP:  13083-859.}

%%
\maketitle

\begin{abstract}
xxx
\end{abstract}

% Head 1
\section{Introduction} \label{prelim}
The study of society's sectors that, due to their high proportion in the whole electoral group, ends deciding countries' elections is always a current topic. Social aspects of the moment and an accurate diagnosis can be identified that explains the victories. 
One aspect that could help in the construction of such a diagnosis is the identification of the relationships that occur between social groups. We investigate the relationship between the number of voters in the winner of the presidential elections in Brazil (2018) and the percentage of individuals who share the predominant religion. 
\setcounter{section}{1}
\section{Data and Models}\label{model}
We approach this problem using the concept of copula, where if $H$ is the cumulative distribution function of $(X,Y)$, there is a function $C,$ such that $H(x,y)=C(F_X(x), F_Y(y)), $  with $F_X(x)=\lim_{y \to \infty}H(x,y)$ and $F_Y(y)=\lim_{x \to \infty}H(x, y).$ And the function $C$ is the 2-copula of $(X,Y).$ $C(u,v)=\mbox{Prob}(F_X(X) \leq u, F_Y(Y) \leq v),$ for $u, v \in [0,1],$ then, $C$ is the distribution of the variables $U:=F_X(X)$ and $V:=F_Y(Y).$ To give flexibility to our analysis we investigated 5 types of dependences, (i) Normal copula, (ii) t-student copula, and 3 Archimedean copulas,  We say that the dependence between $X$ and $Y$ follows an Archimedean copula, generated by $\phi,$ if $(X,Y)$ is a pair of continuous random variables with associated 2-copula $C$ given by the form
$C(u,v )=\phi^{-1}(\phi(u)+\phi(v)) \,\,\,u,v \in [0,1], $
with $\phi(\cdot)$ a continuous, strictly decreasing function from $[0,1]$ to $[0,\infty),$ such that $\phi(1)=0$ and $\phi^{-1}$ is the pseudo-inverse of $\phi, $ which is equal to the usual inverse in $t \in [0,\phi(0)]$ and is equal to zero in $t \in [\phi(0), \infty],$ see \cite{Nelsen2007} for details. The values $u$ and $v$ are related to the random variables $U$ and $V.$ Then, the 3 remaining copulas that we consider are (iii) Gumbel's copula ($\phi(t)=\{-\ln(t)\}^{\theta}, \theta \in [1, \infty) $) , (iv) Joe's copula ($\phi(t)=-\ln\{1-(1-t)^{\theta}\}, \theta \in [1, \infty)$) and (v) Frank's copula ($\phi(t)=-\ln\{\frac{e^{-\theta t}-1}{e^{-\theta}-1}\}, \theta \in (-\infty, \infty) \setminus \{0\}$). See \cite{Nelsen2007} for a description of each family.
\section{Model Selection}
In order to proceed with the estimation of the model, the original observations $\{(x_{i}, y_{i})\}_{i=1}^n$ are replaced by their re-scaled marginal ranks to $[0,1]$ (pseudo-observations), $u_i := \frac{\vert \{j: 1 \leq j \leq n, x_{j} \leq x_{i} \} \vert }{n} $ and $v_i:=\frac{\vert \{j: 1\leq j \leq n, y_{j} \leq y_{i} \} \vert }{n},\,\, i=1,\cdots,n,$ where $|A|$ denotes the cardinal of the set $A.$ Through the pseudo-observations $\{(u_i,v_i)\}_{i=1}^n,$ for copula families (i)-(v) we construct the likelihood 
$\prod_{i=1}^n c (u_i, v_i),$ where $c$ is the density of the copula.\\
In all the cases we use the {\it{Copula R-package}}, and the function {\it{fitCopula()}}, with arguments \textit{copula} (1) and \textit{method} (2), with (1) `normalCopula(dim=2)',  `tCopula(dim=2)',  `frankCopula(dim=2)',  `gumbelCopula(dim=2)', `joeCopula(dim=2)', respectively  and (2) method= `mpl'.

\section{Conclusions}\label{conclu}
\section{Acknowledgement}
\bibliography{<your-bib-database>}

%% Authors are advised to submit their bibtex database files. They are
%% requested to list a bibtex style file in the manuscript if they do
%% not want to use model1-num-names.bst.

%% References without bibTeX database:

 \begin{thebibliography}{00}

\bibitem{cover} T. M. Cover, {\it{Elements of information theory}} (John Wiley \& Sons, 1999).

\bibitem{Huffman} D. A. Huffman, A method for the construction of minimum-redundancy codes, {\it{Proceedings of the IRE}}, Vol. {\bf{40}}(9), 1098-1101 (1952).


%\bibitem{Gustavo2020b} Jes\'{u}s E. Garc\'{\i}a, V.A. Gonz\'alez-L\'opez and G.H. Tasca. A Stochastic Inspection about Genetic Variants of Covid-19 Circulating in Brazil during 2020. {\it{AIP Conference Proceedings}} (forthcoming)

%\bibitem{Gustavo2020}Jes\'{u}s E. Garc\'{\i}a, V.A. Gonz\'alez-L\'opez and G.H. Tasca, Partition Markov Model for Covid-19 Virus. {\it{4open}} {\bf{3}}, 13 (2020).

\bibitem{Garcia2017}  Jes\'{u}s E. Garc\'{\i}a and V. A. Gonz\'alez-L\'opez, Consistent Estimation of Partition Markov Models, Entropy. {\bf{19}}(4), 160 (2017).

%\bibitem{GGGL2018} Jes\'{u}s E. Garc\'{ \i}a, R. Gholizadeh and V.A. Gonz\'{a}lez-L\'{o}pez, A BIC-based consistent metric between Markovian processes. {\it{ Applied Stochastic Models in Business and Industry}}  {\bf{34}}(6), 868-878 (2018) 

\bibitem{Schwarz1978} G. Schwarz, Estimating the dimension of a model, The annals of statistics. {\bf{6}}(2), 461-464 (1978).

%\bibitem{CGGLL2019} M.T.A. Cordeiro,  Jes\'{u}s E. Garc\'{ \i}a, V. A. Gonz\'{a}lez‐L\'{o}pez,  S. L. Mercado Londo\~{n}o, Stochastic profile of Epstein-Barr virus in nasopharyngeal carcinoma settings. {\it{4open}} {\bf{2}}, 25 (2019)

%\bibitem{Cordeiroetal2020} M.T.A. Cordeiro,  Jes\'{u}s E. Garc\'{ \i}a, V. A. Gonz\'{a}lez‐L\'{o}pez,  S. L. Mercado Londo\~{n}o,  Partition Markov model for multiple processes. {\it{Mathematical Methods in the Applied Sciences,}} {\bf{43}}(13), 7677-7691 (2020)
\bibitem[Joe, 1993]{Joe1993} H. Joe, Parametric families of multivariate distributions with given margins. {\it{Journal of multivariate analysis}} {\bf{46}}(2), 262-282 (1993).

\bibitem[Nelsen, 2007]{Nelsen2007} R. B. Nelsen, {\it{An introduction to copulas}} (Springer Science \& Business Media, 2007).

\bibitem[Piketty, 2014]{Piketty2014} T. Piketty, {\it{O capital no s\'{e}culo XXI,}} (Editora Intr\'{ \i}nseca, 2014).

\bibitem[OECD, 2014]{OECD}OECD (2014) {\it{Focus on Top Incomes and Taxation in OECD Countries: Was the Crisis a Game Changer?,}} Directorate for Employment, Labour and Social Affairs Paris.

\bibitem[Genest and Ne{\v{s}}lehov{\'a}, 2012]{genest2012tests} C. Genest and J. Ne{\v{s}}lehov{\'a}, Tests of symmetry for bivariate copulas. {\it{Annals of the Institute of Statistical Mathematics}} {\bf{64}}(4), 811--834 (2012).


  
\end{thebibliography}
\end{document}
