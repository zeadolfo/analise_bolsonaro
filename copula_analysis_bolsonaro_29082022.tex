\documentclass{aip-cp}
%\usepackage{natbib}
\bibliographystyle{aipnum-cp}
%\setcitestyle{square, comma, numbers,sort&compress, super}
\usepackage[T1]{fontenc}
\usepackage{rotating}
\usepackage{url}
\usepackage{graphicx}
\usepackage[dvipsnames]{xcolor}
\usepackage{soul}
\usepackage[utf8]{inputenc}
%\usepackage{mathtools}
%\usepackage{amsmath}

%\bibliographystyle{aip}
%\bibliography{sampleBibFile}
%\usepackage{authblk}


%\newcommand*{\eaddress}[1]{%
%    \normalsize\href{mailto:#1}{#1}\par
%}

%\newcommand*{\affil}[1]{%
%    \normalsize\href{#1}{#1}\par
%}


\newtheorem{definition}{Definition}[section]
\newtheorem{theorem}{Theorem}[section]
\newtheorem{Example}{Example}[section]
\newtheorem{Simulation}{Simulation}[section]


%\renewcommand*{\Authsep}{, }
%\renewcommand*{\Authand}{, }
%\renewcommand*{\Authands}{, }
%\renewcommand*{\Affilfont}{\normalsize\normalfont}
%\renewcommand*{\Authfont}{\bfseries}    % make author names boldface
%\setlength{\affilsep}{2em}   % set the space between author and affiliation

%\newsavebox\affbox

\title{Copula Analysis of the 2018 Brazilian Presidential Election}

%%
\author{J.A.A. Schultz} %\thanks{Corresponding Author: zeadolfo@gmail.com}}
\eaddress{zeadolfo@gmail.com}
%\affil{Agile Thought. Calle Bosque de Alisos. Mexico City, Mexico. CP: 05120.}


\author{V.A. Gonz\'alez-L\'opez} %\thanks{Corresponding Author:veronica@ime.unicamp.br}}
\eaddress{veronica@ime.unicamp.br}
\affil{University of Campinas - Department of Statistics. Rua S\'{e}rgio Buarque de Holanda 651. Campinas, S.P. Brazil. CEP:  13083-859.}



% Document starts
\begin{document}
\date{}


% Title portion

\maketitle


%%


\begin{abstract}
In this article, we stablish the dependence between the percentage of voters in the president of Brazil elected in the second round of 2018 and the percentage of self-declared evangelical voters. Considering the percentages state by state, we show that the dependence between both quantities can be well represented by a Gaussian copula, selected between six possible copulas. We identify the correlation coefficient $\rho$ of the Gaussian copula through a Bayesian approach which allows us to determine a posterior distribution of $\rho,$ with a mean value $ 0.727$ and a 95\% high density credibility interval $[0.531, 0.846].$ 

\end{abstract}

% Head 1
\section{Introduction} \label{prelim}
This work aims to investigate the relation between a specific social group of the population of Brazil and the abrupt change in the political line that governs Brazil since 2018. This research is especially relevant at this time when Brazil faces elections again, in October 2022, having as main actors in the electoral contest, the current president and to what is considered his opposite, the former president Lula da Silva.
Under the democratic system, when drastic changes occur in the politics of a country, it could be a consequence of social changes happens in its population. For example, certain previously smaller groups may have become large enough to allow them to disrupt it previous profile. It appears to have been the situation faced by Brazil in 2018, when, after more than 15 years of center-left governments, it was elected a far-right candidate.  In recent decades, one of the groups that have grown the most in the population of Brazil is evangelicals and it is for this reason that this is the group that we included in our inspection. With this goal, we investigate the relationship between the number
of voters in the winner of the presidential elections in Brazil (2018) and the percentage of individuals who share the
predominant religion in Brazil.

\setcounter{section}{1}
\section{Data and Model Selection}\label{model}

We approach this problem using the concept of copula, since we want to model the relationship between $X$ which is {\it{the proportion of voters in the winner of the presidential elections in Brazil (2018)}} and $Y$ which is {\it{the percentage of individuals who share the predominant religion in Brazil}}. If $H$ is the cumulative distribution function of $(X,Y)$, there is a function $C,$ such that $H(x,y)=C(F_X(x), F_Y(y)), $  with $F_X(x)=\lim_{y \to \infty}H(x,y)$ and $F_Y(y)=\lim_{x \to \infty}H(x, y).$ And the function $C$ is the 2-copula of $(X,Y).$ $C(u,v)=\mbox{Prob}(F_X(X) \leq u, F_Y(Y) \leq v),$ for $u, v \in [0,1],$ then, $C$ is the distribution of the variables $U:=F_X(X)$ and $V:=F_Y(Y).$ To give flexibility to our analysis we investigate 6 types of dependences, (a) Gaussian copula, (b) t-student copula, (c) Frank copula, (d) Gumbel copula, (e) Clayton copula and (f) Joe copula (see \cite{nelsen}). The idea is to include a broad spectrum of dependence types since the t-student copula can fit better tail values and the archemedian copulas can detect better other kind of dependences. Then, $U$ and $V$ are represented by its pseudo-observations which the values $u$ and $v$ are related to the random variables $U$ and $V$, build in our case by the empirical (and) marginal distributions, respectively.

The data is related to the 27 states of Brazil:  Acre (AC), Alagoas (AL), Amapá (AP), Amazonas (AM), Bahia (BA),  Ceará (CE), Distrito Federal (DF), Espírito Santo (ES), Goiás (GO), Maranhão (MA), Minas Gerais (MG), Mato Grosso (MT), Mato Grosso do Sul (MS), Pará (PA), Paraíba (PB), Pernambuco (PE), Paraná (PR), Piauí (PI), Rio de Janeiro (RJ), Rio Grande do Norte (RN), Rio Grande do Sul (RS), Rondônia (RO), Roraima (RR), Santa Catarina (SC),  São Paulo (SP),  Sergipe (SE), Tocantins (TO). The data set related to $Y$ can be retrieved from {\texttt{Instituto Brasileiro de Geografia e Estatística, Censo 2010}}, https://censo2010.ibge.gov.br/apps/mapa \footnote{Last view, 17 of June of 2022}, and the data set related to $X$ is coming from {\texttt{Tribunal Superior Eleitoral (TSE), October 7th 2018}}, 
 https://sig.tse.jus.br/ords/dwapr/seai/r/sig-eleicao-resultados/home?session=6044292194297 \footnote{Last view, 17 of June of 2022}\\

 In order to proceed with the estimation of the model, the original observations $\{(x_{i}, y_{i})\}_{i=1}^n$, where $x_{i}$ is the proportion of evangelicals in the state $i$ and $y_{i}$ is the proportion of votes in Bolsonaro in 2018 in the second round in the state $i$, are replaced by their re-scaled marginal ranks to $[0,1]$ (pseudo-observations), $u_i := \frac{\vert \{j: 1 \leq j \leq n, x_{j} \leq x_{i} \} \vert }{n} $ and $v_i:=\frac{\vert \{j: 1\leq j \leq n, y_{j} \leq y_{i} \} \vert }{n},\,\, i=1,\cdots,n,$ where $|A|$ denotes the cardinal of the set $A,$ and $n=27.$ We see in Figure \ref{graf1} the scatterplot between $X$ and $Y,$ represented by the observations $\{(x_i,y_i)\}_{i=1}^n,$ and in Figure \ref{graf2} the scatterplot between $U$ and $V,$ represented by the pseudo-observations $\{(u_i,v_i)\}_{i=1}^n.$
%We can note in the Figure~\ref{graf1} and Figure~\ref{graf2} the lack of extreme values and no right or left tail dependence which favored the Gaussian and the Frank copulae and the Table~\ref{tabla1}. 
\begin{figure}[!ht]
\caption{Scatterplot between the proportion of evangelicals ($x$) and proportion of votes for Bolsonaro ($y$), state by state.} \label{graf1}
\centering
\includegraphics[scale=.6]{"prop_evang_votes.jpeg"}
\end{figure}
\begin{figure}[!ht]
\caption{Scatterplot between pseudo-observations: $u$ versus $v$, state by state.} \label{graf2}
\centering
\includegraphics[scale=.6]{"empirical dist envag votes.jpeg"}
\end{figure}
Such figures (\ref{graf1}, \ref{graf2}) support the idea that there is some dependence between $X$ and $Y$, which becomes the focus of our investigation in the following lines. The Spearman's correlation coefficient given by the data is 0.7232, confirming the tendency exposed by the data in figure \ref{graf2}.\\
%\footnotetext[1]{AC: Acre, AL: Alagoas, AP: Amapá, AM: Amazonas, BA: Bahia, CE: Ceará, DF: Distrito Federal, ES: Espírito Santo, GO: Goiás, MA: Maranhão, MG: Minas Gerais, MT: Mato Grosso, MS: Mato Grosso do Sul, PA: Pará, PB: Paraíba, PE: Pernambuco, PR: Paraná, PI: Piauí, RJ: Rio de Janeiro, RN: Rio Grande do Norte, RS: Rio Grande do Sul, RO: Rondônia, RR: Roraima, SC: Santa Catarina, SP: São Paulo, SE: Sergipe, TO: Tocantins}
Through the pseudo-observations $\{(u_i,v_i)\}_{i=1}^n,$ for copula families (a)-(f) we construct the likelihood  $\prod_{i=1}^n c (u_i, v_i),$ where $c$ is the density of the copula, in each case.
In all the cases we use the {\it{Copula R-package}}, and the function {\it{fitCopula()}}, with arguments \textit{copula} (1) and \textit{method} (2), with (1) `frankCopula(dim=2)', `normalCopula(dim=2)',  'gumbelCopula(dim=2)', 'claytonCopula(dim=2)', `tCopula(dim=2, t = 1)',  `tCopula(dim=2, t = 10)',   'joeCopula(dim=2)', respectively  and (2) method= `mpl'.
The package also provides a goodness-of-fit test (see \cite{genest}) and it uses the maximum pseudo likelihood estimator to estimate the parameter for each copula. We report the results in Table~\ref{tabla1}. The power of this test decreases as the sample size decreases. However, it is not affect the results here since we have a lot of evidence in favor of the most of copulas.
%
\begin{center}
\begin{table}[!h]
\caption{Goodness-of-fit for copula. In bold the indicated model.}
\label{tabla1}
\centering
\begin{tabular}{c|c|c|c|c|c|c|c}
Copula&  {\bf{Gaussian}} & Frank &  t(10) & t(1) &Clayton & Gumbel & Joe \\
P-value&  {\bf{0.934}} & 0.922 & 0.844  &  0.829 & 0.749 & 0.512 & 0.041
\end{tabular}
\end{table}
\end{center}

%
Given the results of Table \ref{tabla1}, the Gaussian copula is the most suitable to represent the dependence between $X$ and $Y.$ Given the reduced number of data, $n=27,$ we implement a Bayesian approach to determine the correlation of the copula. This approach provides a flexible inspection, as we show in the next section.\\

\section{Bayesian Inference on the Correlation Coefficient}\label{bayesian}
Once fixed the Gaussian copula, we provide a Bayesian inspection of the correlation coefficient $\rho,$ which is the copula's parameter in the case.  For the study, we choose three prior distributions on $\rho$ building three settings for $\rho$, (i) an impartial setting, with prior $\rho \sim$ Beta(1,1), (ii) partial setting, with prior $\rho \sim$ Beta(1,15), favoring negative dependence, (iii) a partial setting, with prior $\rho \sim$ Beta(15,1), favoring positive dependence.
Note that since $\rho \in [-1,1]$ the Beta distributions stated in i, ii and iii are Beta displayed in $[-1,1]$ instead of $[0,1].$\\
Now we introduce briefly the Full Bayesian Significance Test (FBST), used as an alternative option to decide if we have or not significant evidence against the relation between $X$ and $Y.$ This is a test specially designed for a Bayesian context, see \cite{pereira}, \cite{madruga}. Let $\pi\Big(\rho \vert \{(u_i,v_i)\}_{i=1}^n\Big)$ be the 
posterior distribution of $\rho$ given $\{(u_i,v_i)\}_{i=1}^n$ and let be an evidence measure in $H_0: \rho \in \Theta_0$ using the tangent set called $T,$
with the more ``probable'' points for the set $\Theta_0 ,$
\begin{center}
 $T = \Big\{\rho \in \Theta: \pi\Big(\rho| \{(u_i,v_i)\}_{i=1}^n\Big) > t\Big\}$ where $t = \sup_{\rho \in \Theta_0}\pi\Big(\rho\vert \{(u_i,v_i)\}_{i=1}^n \Big).$ 
\end{center}
Then, the evidence $e$-value in favor of the set $\Theta_0$ (see \cite{pereira}) is given by
$e$-value = $1 - \mathbb{P}\Big(\rho \in T| \{(u_i,v_i)\}_{i=1}^n \Big).$
Small $e$-values point the rejection of $H_0.$\\
We estimate the posterior distributions (by Hamiltonian Monte-Carlo), under the settings (i), (ii) and (iii). We also report the $e$-value for $\Theta_0=\{0\}$ (null coefficient), to decide if the hypothesis of independence between $X$ and $Y$ is feasible.  The results are reported in Table~\ref{tabla2}. 
\begin{center}
\begin{table}[h!]
\label{tabla2}
\caption{Summary measures for posterior distributions and independence test e-value}
\centering
\begin{tabular}{c|c|c|c|c|c|c|c|c|c}
Case&Prior & Mín & 1st Qu. & Median & 3rd Qu. & Max & Mean & Std Dev & $e$-value \\
\hline 
(i)&Beta(1,1) & 0.173 & 0.684 & 0.742 & 0.785 & 0.896 & 0.727 & 0.083 & < 0.001\\
(ii)&Beta(1,15) & -0.471 & -0.048 & 0.063 & 0.187 & 0.712 & 0.072 & 0.172 & 0.353 \\
(iii)&Beta(15,1) & 0.381 & 0.735 & 0.775 & 0.811 & 0.892 & 0.769 & 0.060 & < 0.001
\end{tabular}
\end{table}
\end{center}
%The posterior mean using the prior distribution Beta(1,1) is 0.7268, that value is very close to the Spearman's correlation given by the data, 0.7232 and rejects $H_0: \rho = 0$ ($e$-$value < 0.001$). The posterior distribution, using as a prior distribution on $\rho$, the Beta(15,1), as expected, rejects $H_0: \rho = 0$ ($e$-$value < 0.001$) . On the other hand, using a prior as Beta(1,15), the hypothesis $H_0: \rho = 0$ is not rejected ($e$-$value = 0.353$) since the mixture between the prior distribution and the likelihood function brings the posterior distribution around the 0 correlation. This fact demonstrates the model is not robust as we can observe in Figure~\ref{graf3}, when the prior distribution favoring the negative dependence pushes the posterior distribution to have a median position near zero. The selection of a prior distribution affects the final inference mainly if the prior distribution is elicitated against the data tendency.
\begin{figure}[h!]
\caption{Histograms of posterior distributions with prior distributions Beta on $\rho$ (in red). From left to right: setting i. (using a prior Beta(1,1)), setting ii. (using a prior Beta(1,15)) and setting iii. (using a prior Beta(15,1)). } \label{graf3}
\centering
\includegraphics[scale=.7]{"gaussian_priors.jpg"}
\end{figure}
Starting from the fact that the Spearman coefficient found in the data is 0.7232, cases (i) and (iii) are the most reasonable. Case (ii) (favorable to negative $\rho$ values) is included to verify how it rearranges the $\rho$ values when there is a conflict between the prior information and the likelihood function. (i) is adequate when there is no preference for any value of $\rho$ in the support [-1,1], and case (iii) is adequate when the prior information tends to positive values of $\rho .$ In such scenarios, the assumption of independence ($\rho=0$) must be rejected since the $e$-values take negligible values (see the last column of table \ref{tabla2}). We note that the impartial setting (i)  indicates a Bayesian estimator, by quadratic loss function, equal to 0.727, that is, slightly higher than the observed correlation, while the most favorable scenario for positive correlation, (iii) gives a Bayesian estimator equal to 0.769, showing the effect of the prior distribution. We see in figure \ref{graf3}, when comparing (iii) with (i), that the posterior distribution of $\rho$ suffers a (shy) reduction in its precision, 3rdQu-1stQu goes from 0.101 (in (i)) to  0.076 (in (iii)). On the other hand,  for setting ii, the hypothesis $H_0: \rho = 0$ is not rejected with $e$-value = 0.353. Moreover, the combination of the prior distribution (Beta(1,15)) and the likelihood function brings the posterior distribution around $\rho=0$ (see figure \ref{graf3}-middle) showing a clear conflict between the data and the prior distribution.

%In the Figure~\ref{graf3}, we have the posterior histogram, the prior density (red line) and in the point zero the point where the correlation is zero. In our case, independence since we are using the Gaussian copula.   

%The posterior mean using the prior distribution Beta(1,1) is 0.7268, that value is very close to the Spearman's correlation given by the data, 0.7232 and rejects $H_0: \rho = 0$ (e-value $< 0.001$). The posterior distribution, using as a prior distribution on $\rho$, the Beta(15,1), as expected, rejects $H_0: \rho = 0$ (e-value $< 0.001$) . On the other hand, using a prior as Beta(1,15), the hypothesis $H_0: \rho = 0$ is not rejected e-value $= 0.353$) since the mixture between the prior distribution and the likelihood function brings the posterior distribution around the 0 correlation. This fact demonstrates the model is not robust as we can observe in Figure~\ref{graf3}, when the prior distribution favoring the negative dependence pushes the posterior distribution to have a median position near zero. The selection of a prior distribution affects the final inference mainly if the prior distribution is elicitated against the data tendency.

%We can note, moreover, the selection of prior distributions affected the final inferences mainly the prior distribution against the data.

\section{Conclusions}\label{conclu}
In this work we identify the dependence between $X$ which is {\it{the proportion of voters in the winner of the presidential elections in Brazil (2018)}} and $Y$ which is {\it{the percentage of individuals who share the predominant religion in Brazil}}. We show that the dependence can be well represented by a Gaussian copula. To select the copula we use the test introduced in \cite{genest} (see table \ref{tabla1}). In a second stage of this work, since the database is relatively moderate ($n=27$), we conduct a Bayesian estimation of the Gaussian copula's correlation coefficient, see table \ref{tabla2} and figure \ref{graf3}. In this implementation we consider 3 settings that allow us to compare the effect of 3 prior distributions on the $\rho$ correlation coefficient. The non-informative scenario (setting (i)) is taken as a reference and we confirm the non-nullity of such coefficient ($\rho$) by means of a genuinely Bayesian independence test, with $e$-value<0.001 (see \cite{pereira}, \cite{madruga}). Also, this scenario gives a Bayesian estimator of $\rho= 0.727$  with a 95\% high density credibility interval $[0.531, 0.846],$ then, we obtain a range of values for $\rho$ confirming a non-irrelevant relation between $X$ and $Y.$ For the future studies, we analyze the dependence between evangelicals and the candidates elected in 2022 and the new Brazilian census.


%to investigate the dependence between the percentage of voters in the president of Brazil elected in the second round of 2018 and the percentage of self-declared evangelical voters


%For future studies, to make the application really useful, it would be interesting to analyze the dependence between this social group and the candidates elected in 2010 and 2014 since the data is from 2010.





%In this paper, we explore the using of Full Bayesian Statistic Test in election scenario to 
%understand the voters behaviour in the second round in 2018 in Brazil election using copula models. 
%Using the Genest's test (see \cite{genest}), we find the Gaussian copula as the best model, this copula 
%offers us interpretability for the dependence parameter since it is the correlation, i.e., the parameter $\rho = 0$ means independence. 
%
%The correlation between the proportion of evangelical people is highly correlated with the proportion of votes in each Brazil state with $\rho$ mean $0.727$ in the scenario with the prior density Beta(1,1), $95\%$ credibility interval $[0.531 0.846]$ which means the Bolsonaro's evagenlical bases is preponderant. Using the FBST, we reject the hypothesis $H_0: \rho = 0$ with a $e$-$value$ very closer to 0.
%This results are similar with the frequentist interpretation. However, without the flexility and straitforward interpretation.
%
%We can see also the posterior density are influenced by the prior since the prior against the data, density Beta(1,15), push the posterior density to 1 and the prior favor the data, density Beta(15,1), changed the conclusion no rejecting the independence hipothesis with $e$-$value = 0.353$
%
%Interesting future works are explore others variables as agrobusiness indicators or guns per capita by state and create a multivariate model.


%\section{Acknowledgement}
%\bibliography{<your-bib-database>}

%% Authors are advised to submit their bibtex database files. They are
%% requested to list a bibtex style file in the manuscript if they do
%% not want to use model1-num-names.bst.

%% References without bibTeX database:

\begin{thebibliography}{20}

%\bibitem{censo} {\color{red}Census (Instituto Brasileiro de Geografia e Estatística, 2010)}
%
%\bibitem{tse} {\color{red}TSE - Election Results (Tribunal Superior Eleitoral, October 7th 2018)}

%\bibitem{joe} H., Joe, \textit{Dependence Modeling with Copulas} (Taylor \& Francis Group, Boca Raton, 2014)

\bibitem{nelsen} R. B., Nelsen, \textit{An introduction to copulas} (Springer, New York, 2007).
%
\bibitem{genest} C., Genest,  B., R\'{e}millard and D., Beaudoin, Goodness-of-fit tests for copulas: a review and a power study, {\it{Insurance: Mathematics and Economics}} \textbf{44}(2), 199-213 (2009). https://doi.org/10.1016/j.insmatheco.2007.10.005
%
\bibitem{pereira} C.A.B., Pereira and J.M., Stern, Evidence and Credibility: Full Bayesian Significance Test for Precise Hypotheses. {\it{Entropy}} \textbf{1}(4), 99-110 (1999). https://doi.org/10.3390/e1040099
%
\bibitem{madruga} M.R., Madruga, L.G., Esteves and S., Wechler, On the bayesianity of Pereira-Stern Tests. {\it{Test}} \textbf{10}(2), 291-299 (2001). https://doi.org/10.1007/BF02595698


%\bibitem{congdon} P. Congdon, \textit{Applied Bayesian Modeling} (John Wiley \& sons, West Sussex, 2014).

%\bibitem{vuolo} M. Vuolo, Copula Models for Sociology: Measures of Dependence and Probabilities for Joint Distributions, Sociological Methods \& Research \textbf{46}, 604-648 (2015).


%\bibitem[Joe, 1993]{Joe1993} H. Joe, Parametric families of multivariate distributions with given margins. {\it{Journal of multivariate analysis}} {\bf{46}}(2), 262-282 (1993).

%\bibitem[Nelsen, 2007]{Nelsen2007} R. B. Nelsen, {\it{An introduction to copulas}} (Springer Science \& Business Media, 2007).





\end{thebibliography}
\end{document}
